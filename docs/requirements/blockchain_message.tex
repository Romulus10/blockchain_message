\documentclass[titlepage]{report}

\usepackage[toc,page]{appendix}
\usepackage{glossaries}
\usepackage{makeidx}
\usepackage{biblatex}
\usepackage{graphicx}
\usepackage{float}

\makeglossaries{}
\loadglsentries{entries}

\makeindex
\addbibresource{bib.bib}

\title{blockchain\_message: A Peer-to-Peer Content-Sharing Service Built on the Ethereum Blockchain\\\large System Requirements Specification}
\author{Sean T. Batzel\\Dr.\ Bishop}
\date{\today\endgraf\bigskip Submitted in partial fulfillment of the requirements of CMPS/IT 490 --- Computer Projects}

\begin{document}
\maketitle

\nocite{*}

\section{Introduction}

\section{System Model}

\section{Functional Requirements}
\subsection{General Features Overview}
\subsection{Security and Privacy}

\section{User Interface Specification}
The UI will have three levels during development, starting with a command-line interface, with a median goal being an HTML5-based graphical interface and the ultimate goal being a native Android application.

\subsection{Command-Line Interface}
The CLI will accept a few simple commands for interacting with the blockchain\_message system. On running the program, you will be met by a login screen to prompt for a user address, a username, and an email address. These three elements will be used to identify you to the network while exchanging messages.
The interface commands are as follows:

\texttt{write}

\texttt{check}

\texttt{read}

\texttt{new-contact}

\texttt{exit}

\subsection{HTML5 Interface}
The HTML5-based interface will expose all of the same functionality as the command-line interface.

\subsection{Android Application}

\section{Non-functional Requirements}
\subsection{Software Requirements}
The first version of the blockchain\_message command-line interface is designed to run on Ubuntu Linux\index{Linux}, but with tweaking can be made to run on any system with Python installed. The installation script can be easily modified for installation on any Linux distribution.

The system requires that an Ethereum\index{Ethereum} \gls{node}\index{node} be accessible, which can be either set up on the local machine through Parity or Geth or exposed through a service such as Infura.io.

The latest version of Python (at the time of writing, 3.6.6 is installed locally) is required, as well a list of Python packages that are installed by the \texttt{install} script.

\subsection{Hardware Requirements}
\paragraph{If Using a Local Ethereum Node}
If you're using a node\index{node} set up on another system, ignore this paragraph. If Parity, Geth, or a similar Ethereum node is running on the local machine, blockchain\_message requires available resources for both the appliation itself and the Ethereum node. The entire Ethereum blockchain\index{blockchain} will require approximately 300 gigabytes of disk space and 4 gigabyte of available memory. Usually, it's preferable to have 2 CPU cores for blockchain\index{blockchain} operations

\paragraph{blockchain\_message Requirements}
The blockchain\_message process itself will require 250 megabytes of available disk space.

\subsection{Product Standards}
\subsubsection{User Manual}
\subsubsection{Programming Manual}

\subsection{Process Standards}
\subsubsection{Code Quality/Convention}
\subsubsection{Comment Quality/Convention}

\section{System Evolution}
\subsection{Cross-Platform Support}

\printindex
\printglossaries{}
\printbibliography{}

\end{document}
